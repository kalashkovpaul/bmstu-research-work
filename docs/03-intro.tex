\section*{ВВЕДЕНИЕ}
\addcontentsline{toc}{section}{ВВЕДЕНИЕ}

Работа с гипертекстовыми документами является неотъемлемой частью жизни каждого человека, пользующегося Всемирной сетью.
Часто возникает потребность в просмотре различных гипертекстовых документов, а также в выполнении операций, приводящих к их изменению.
Возникает вопрос: каким образом стоит отображать документ, и производить операции его обновления и построения?

Для взаимодействия с гипертекстовыми документами, входящими в сеть Интернет, существуют программы-браузеры.
Преимущественнная часть браузеров использует стандарт \cite{dom-doc}, обеспечивающий использование объектной модели документа \cite{dom}. 


\textbf{Целью данной работы} является анализ алгоритмов построения, обноовления и отображения гипертекстового документа при помощи виртуальной объектной модели.
Для достижения поставленной цели необходимо выполнить следующие задачи:

\begin{enumerate}[label=\arabic*)]
	\item изучить принципы работы объектной модели документа;
	\item изучить принципы работы виртуальной объектной модели документа;
	\item сравнить и проанализировать трудоёмкости алгоритмов с использованием объектной модели документа и виртуальной объектной модели документа на основе теоретических расчётов.
\end{enumerate}


\pagebreak