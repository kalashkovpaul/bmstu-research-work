\section*{ЗАКЛЮЧЕНИЕ}
\addcontentsline{toc}{section}{ЗАКЛЮЧЕНИЕ}

Была достингута цель работы: проанализированы алгоритмы построения, обновления и отображения гипертекстового документа при помощи объектной модели и виртуальной объектной модели.
Также в ходе выполнения научно-исследовательской работы были решены следующие задачи: 
\begin{enumerate}[label=\arabic*)]
	\item изучены принципы работы объектной модели документа и работы виртуальной объектной модели документа;
	\item проведены сравнение и анализ трудоёмкостей алгоритмов обновления документа с использованием объектной модели и виртуальной объектной модели на основе теоретических расчётов;
	\item сделаны выводы об эффективности использования изученных алгоритмов.
\end{enumerate}

Исходя из полученных результатов, трудоёмкость алгоритма обновления документа с использованием DOM пропорциональна $\Theta(xn)$, где $x$ --- трудоёмкость операции update отрисовки узла, а $n$ --- количество узлов в новом DOM-дереве.
В это время трудоёмкость обновления документа с использованием VDOM и алгоритма согласования в случае соблюдения предположений, лежащих в основе алгоритма согласования, пропорциональна $\Theta(xk)$, где $k$ --- количество узлов, требующих перерисовки, $k << n$ в общем случае.

Таким образом, алгоритм обновления гипертекстового документа с использованием VDOM и алгоритма согласования имеет меньшую трудоёмкость при соблюдении эвриситки агоритма согласования, за счёт перерисовки только необходимых улов.
В противном случае, когда эвристика алгоритма согласования не соблюдена, использование VDOM может иметь большую трудоёмкость, чем просто использование DOM.

\pagebreak