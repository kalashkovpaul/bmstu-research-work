\section*{ЗАКЛЮЧЕНИЕ}
\addcontentsline{toc}{section}{ЗАКЛЮЧЕНИЕ}

Была достингута цель работы: проанализированы алгоритмы построения, обновления и отображения гипертекстового документа при помощи объектной модели и виртуальной объектной модели.
Также в ходе выполнения научно-исследовательской работы были решены следующие задачи: 
\begin{enumerate}[label=\arabic*)]
	\item были изучены принципы работы объектной модели документа;
	\item были изучены принципы работы виртуальной объектной модели документа;
	\item были проведены сравнение и анализ трудоёмкостей алгоритмов с использованием объектной модели документа и виртуальной объектной модели документа на основе теоретических расчётов.
\end{enumerate}

Исходя из полученных результатов, алгоритм обновления гипертекстового документа с использованием VDOM и алгоритма согласования имеет меньшую трудоёмкость при соблюдении эвриситки агоритма согласования, за счёт перерисовки только необходимых улов.
В противном случае, когда эвристика алгоритма согласования не соблюдена, использование VDOM может иметь большую трудоёмкость, чем просто использование DOM.

\pagebreak